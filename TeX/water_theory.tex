\section{Введение}
В настоящее время, существующая измерительная аппаратура не всегда позволяет получить достаточно полное представление о состоянии приповерхностного  слоя океана, поэтому постоянно разрабатываются новые радиолокационные системы. 
Вместе с тем, для решения таких задач, как проверка качества диагностики состояния поверхности океана существующими радиолокаторами, тестирование и разработка алгоритмов восстановления океанографической информации, а также оценка возможностей новых радиолокаторов, вполне естественным является применение более экономных по времени и средствам методов, в частности численного моделирования.  Однако, при моделировании одномерной морской поверхности, как правило, используется сумма большого числа гармоник, что приводит к значительным затратам машинного времени.

В связи с этим возникает необходимость в минимизации числа гармоник в спектре моделируемой морской поверхности при сохранении необходимой точности при решении различных задач оптики морской поверхности. Здесь возникает ряд нетривиальных вопросов об оптимальном разбиении частотной плоскости на участки и выборе оптимального положения дискретных спектральных компонент в пределах этих участков. Поиску ответов на эти вопросы и посвящена данная работа.


	Статистическая оптика морской поверхности в настоящее время представляет собой самостоятельную и бурно развивающуюся область оптики моря. Она включает  в себя ряд важных и интересных проблем, основной из которых является изучение статистических характеристик над- и подводных световых полей, созданных как естественными, так и искуственными источниками при наличии случайно-неровной границы раздела воздух-вода. 

	Разработанные к настоящему времени теоретические методы расчета световых полей в море при наличии ветрового волнения на поверхности в основном строятся на геометроортическом описании распространения световых лучей и использовании результатов решения уравнения переноса излучения в рассеивающих средах. Эти методы, как правило, позволяют исследовать статистические моменты лишь первого и второго порядков, причём вычисление последних часто наталкивается, особенно при учёте эффектов двукратного прохождения излучения через морскую поверхность , на существенные и даже порой непреодолимые трудности. Это заставляет исследователей идти по пути упрощения или идеализации как моделей формирования световых полей, так и модели волнения на поверхности раздела. Однако, результаты , полученные в рамках приближенных моделей, могут оказаться не вполне соответствующими действительности и поэтому нуждаются в проверке.



% \section{Цель}

% \section{Формулы для спектров и корреляционных функция волнения}
\section{Теоретическая часть}

\subsection{Спектральное разложение случайных функций}
\begin{equation}
	\label{eq:40.1}
	\zeta(t)= \int\limits_{\infty}^{\infty} e^{i \omega t} \dd{C(\omega)},
\end{equation}
где $\dd{C(\omega)}$ - конечное или бесконечно малое приращение функции $C(\omega)$ на интервале $(\omega, \omega+ \dd{\omega})$:
\begin{equation}
	\dd{C(\omega)}=C(\omega+\dd{\omega})-C(\omega).
\end{equation}
Интеграл Фурье-Стилтьеса \eqref{eq:40.1} охватывает случаи и непрерывного, и дискретного, и смешанного спектров. В чисто дискретном случае можно записать его в форму обобщенного ряда Фурье:
\begin{equation}
	\zeta(t)=\sum\limits_n c_n e^{i \omega_n t},
\end{equation}
а при непрерывном спектре можно формально ввести плотность комплексной амплитуды $\dd{C(\omega)}$:
\begin{equation}
	\dd{C(\omega)}= c(\omega) \dd{\omega},
\end{equation}
так что \eqref{eq:40.1} примет вид обычного интеграла Фурье:
\begin{equation}
	\zeta(t)=\int\limits_{-\infty}^{\infty} e^{i \omega t} c(\omega) \dd{\omega}.
\end{equation}
Если $\zeta$-- случайная функция, то существование интеграла \eqref{eq:40.1}
надо понимать в смысле вероятностной сходимости. Случайные функции, представимые в виде 
\eqref{eq:40.1}, называются гармонизуемыми.

В пределах корреляционной теории, ограничивающейся моментами не выше второго порядка, естественно и целесообразно понимать существование интеграла \eqref{eq:40.1} в среднем квадратичном. Необходимым и достаточным условием гармонизуемости случайной функции $\zeta(t)$ является тогда при любых $ t$ и $t'$ двукратного интеграла Фурье-Стилтьеса
\begin{equation}
	\label{eq:40.7}
	B(t,t')= \int\limits_{-\infty}^{\infty} e^{i(\omega t- \omega't')}\dd[2]{\Gamma}(\omega,\omega')
\end{equation}
представляющего момент второго порядка $B(t,t')= \mean{\zeta(t)\zeta^*(t')} $ функции $\zeta(t)$. Иначе говоря, если интеграл \eqref{eq:40.7} существует, то это означает, что существует случайная комплексная функция $C(\omega)$ такая, что интеграл \eqref{eq:40.1} сходится в среднем квадратичном к $\xi$, причем двумерное приращение функции $\Gamma(\omega, \omega ')$ есть
\begin{equation}
	\dd[2]{\Gamma}=\mean{\dd{C(\omega)}\dd{C^*(\omega')}}.
\end{equation}
В частности, при $t'=t$ имеем
\begin{equation}
\mean{\abs{\zeta(t)}^2}=B(t,t)=\int\limits_{-\infty}^{\infty} e^{i(\omega- \omega')t}\dd[2]{\Gamma}(\omega,\omega').
\end{equation}
Пусть $\zeta(t)$ в широком смысле стационарна. Тогда требование постоянства среднего значения $\mean{\zeta(t)}=\const$ означает, согласно \eqref{eq:40.1}, что при всех $\omega\neq0$ должно быть 
\begin{equation}
 \label{eq:41.1}
 	\mean{\dd{C}(\omega)}=0
 \end{equation} 
 и тогда 
 \begin{equation}
 	\mean{\zeta}=\mean{\dd{C}(0)}.
 \end{equation}
Далее мы будем предполагать, что у рассматриваемых стационарных функций $\zeta(t)$ среднее значение $\mean{\zeta}=0$ и, следовательно, смешанный момент совпадает с функцией корреляции $\psi$.

Условие стационарности функции корреляции может быть выполнено, как это видно из \eqref{eq:40.7}, только в том случае, если
\begin{equation}
	\dd[2]{\Gamma(\omega,\omega')}=\mean{\dd{C(\omega)}\dd{C^*(\omega')}}=0 \text{ при }
	\omega\neq \omega',
\end{equation}
т.е. <<масса>> распределена только на биссектрисе $\omega=\omega'$. Тогда приращение $\dd[2]{\Gamma(\omega,\omega')}$ всегда вещественно и неотрицательно. Если воспользоваться дельта-функцией, то сказанное можно записать в виде
\begin{equation}
	\label{eq:41.2}
 	\dd[2]{\Gamma(\omega,\omega')}=\mean{\dd{C(\omega)}\dd{C^*(\omega')}}=\delta(\omega-
 	\omega')\dd{\omega '}\dd{G(\omega)},
 \end{equation} 
 причем вещественное приращение $\dd{G(\omega)}$ неотрицательно.

Подставив \eqref{eq:41.2}  в \eqref{eq:40.7}, получаем:

\begin{gather}
	\iint\limits_{\infty} e^{i(\omega t - \omega't')} \mean{\dd{C{\omega}}\dd{C^*(\omega')}}= 
	\int\limits_{-\infty}^{\infty}\dd{G(\omega)}
	\int\limits_{-\infty}^{\infty} e^{i(\omega t - \omega't')} \delta(\omega-\omega')\dd{\omega'}=\\
	 \int\limits_{-\infty}^{\infty} e^{i \omega (t-t')} \dd{G(\omega)}=\psi(t-t')
\end{gather}

% Рассмотрим ряд общих понятий, описывающих возвышения и уклоны взволнованной морской поверхности в рамках теории случайных пространственно-временных полей. Представим возвышения поверхности в виде суммы гармонических бегущих волн с независимыми (случайными) фазами:
% \begin{equation}
% 	\label{eq:1}
% 	\zeta(\vec r, t)=\Re \iint\limits_{\infty} \dd{\dot \zeta(k)} \exp{i(\vec k \vec r - \omega_k t)}, 
% \end{equation}
% где $\dd{\dot \zeta}$ -- комплексная амплитуда гармоники с волновым числом $\vec k$  и временной частотой $\omega_k$, связанной с $\vec k$ дисперсионным соотношением $\omega_k=\sqrt{g k }$, $\vec{k}=\vec{x_0} k_x + \vec{y_0} k_y$, $\rho=\vec{x_0} x + \vec{x_0} y$,
% $x_0,y_0$ -- орты декартовой системы координат, $k=\abs{k}$ -- пространственная частота, $t$ -- время, $g=9.81 ~ \frac{\text{м}}{\text{с}^2}$ -- ускорение свободного падения.


\subsection{Небольшое введение в корреляционную теорию}
Докажем стационарность этого процесса в выбранной нами точке $(x_0,y_0)$
\begin{gather}
	\mean { \zeta(x_0,y_0,t) } = 
	\mean{A}\cdot \mean{\cos(\vec k \vec r -\omega t +\theta)} = \\
	\mean{A} \qty( \cos(\vec k \vec r -\omega t) \mean{\cos\theta}
		- \sin(\vec k \vec r -\omega t)\mean{\sin{\theta}} )
\end{gather}
Независимость от $\mean{\zeta(t)}$ от $t$, т.е. равенство $\mean{\zeta (t)=0}$, можно обеспечить при  $\mean {A}=0$ или при $\mean{\cos\theta}=\mean{\sin\theta}=0$. Первый случай нам не подходит, потому рассмотрим второй. Это равенство будет иметь место, если плотность вероятности фазы $w_{\theta}(\theta)$ ортогональна в интервале $(0, 2\pi)$
к $\cos\theta$ и $\sin\theta$, т.е. представима рядом Фурье, но в данной работе будет использоваться частный случай распределения $w_{\theta}(\theta)=\frac{1}{2\pi}$.

*спустя ещё несколько строк рассуждений*
Таким образом , что случайная функция $\zeta$ является стационарной по Хинчкину, то есть её среднее значение постоянно $\zeta(t)=\mean \zeta = \const$,  а момент второго порядка зависит только от $\tau=t_2-t_1$ и конечен при $\tau=0$. 

Пространственно-временная корреляционная
\footnote{Нужно разобраться с названиями, но судя по всему так далее называется смешанный момент}
 функция возвышений по определяется выражением:
\begin{gather}
\label{eq:2}
	M_{\zeta}(\vec{r_1},\vec{r_2},t_1,t_2)= \mean{\zeta(\vec{r_1},t_1)\zeta(\vec{r_2},t_2) }
\end{gather}
В соответствии с \eqref{eq:1}:
\begin{gather*}
	M_{\zeta}(\vec{r_1},\vec{r_2},t_1,t_2)= \frac12 \Re 
	\iint\limits_{\infty}  \iint\limits_{\infty} 
	\mean{\dd{\dot \zeta(\vec k_1)} \dd{\dot \zeta(\vec k_2)} } 
	\exp{i(\vec k_1 \vec r - \omega_1 t_1 + \vec k_2 \vec r - \omega_1 t_2)} +\\ 
	+\mean{\dd{\dot \zeta(\vec k_1)} \dd{\dot \zeta^*(\vec k_2)} } 
	\exp{i(\vec k_1 \vec r - \omega_1 t_1 - \vec k_2 \vec r + \omega_1 t_2)}
\end{gather*}

Поскольку двумерная плотность вероятности стационарного процесса зависит от $t_1$ и $t_2$ через разность $\tau=t_2-t_1$, то смешанный момент второго порядка будет зависеть только от $\tau$\footnote{доказать}. Аналогично можно сказать и про $\vec r_1$ и $\vec r_2)2$

Итак, для статически однородного и стационарного поля выполняется соотношение: 
\begin{equation}
	M_{\zeta}(\vec r_1, \vec r_2,t_1,t_2)=M_{\zeta}(\vec \rho= \vec r_2 - \vec r_1, \tau =t_2-t_1)
\end{equation}
Чтобы это соотношение было справедливым в нашей задаче, необходимо потребовать выполнение условий
\begin{equation}
	\frac12 \mean{\dd{\dot \zeta(\vec k_1)} \dd{\dot \zeta(\vec k_2)} } =0 
	~~~\text{ и }~~~ \frac12 \mean{\dd{\dot \zeta(\vec k_1)} \dd{\dot \zeta^*(\vec k_2)} } =
	\tilde S(\vec k_1) \delta(\vec k_2- \vec k_1) \dd{\vec k_1} \dd{\vec{k_2}}.
\end{equation}
где $\tilde S(\vec k)$ -- волновой спектр морской поверхности, $\delta(\vec k_2 - \vec k_1)$ -- дельта-функция. Подставляя эти условия в \eqref{eq:2}, получим:
\begin{equation}
	\label{eq:3}
	M_{\zeta}(\rho, \tau)= \iint\limits_{\infty} S(\vec{k}) \cos(\vec k \vec \rho - \omega_k \tau) \dd{k}.
\end{equation}

Винеровский энергетический спектр определяется преобразованием Винера-Хинчкина функцией корреляции, описываемой \eqref{eq:3}

\begin{gather}
	\label{eq:4}
	\Phi_{\zeta}(\vec k ,\omega)= \iiint\limits_{\infty} 
	M_{\zeta}(\vec \rho, \tau) e^{-i\qty(\vec k \vec \rho + \omega t)} \dd{\rho} \dd{\tau}=4\pi^3 \qty[\tilde S(\vec k) \delta(\omega+\omega_k)+\tilde S(- \vec k) \delta(\omega- \omega_k)].
\end{gather}
Из \eqref{eq:4} следуют, как частные случаи, выражения для пространственного,
\begin{equation}
	\Phi_{\zeta}(\vec k)=\frac{1}{2\pi} \int\limits_{\infty} \Phi_{\zeta}(\vec k, \omega) \dd{\omega}= 2 \pi^2 \qty[\tilde S(\vec k)+\tilde S(-\vec k)],
\end{equation}
и временного спектров Винера.

{\color{red}{Разобрать фундаментально всю  теорию до этого момента.}}